\documentclass[10pt,a4paper]{book}
\usepackage[utf8x]{inputenc}
\usepackage{ucs}
\usepackage[italian,english]{babel}
\usepackage{amsmath}
\usepackage{amsfonts}
\usepackage{amssymb}
\usepackage{graphicx}
\author{Wahid}
\title{Machine Learning}

\begin{document}
	
\chapter{Generazione numeri casuali sotto BW}
Si può fare una differenziazione iniziale:
\begin{itemize}
	\item Numeri casuali: truly random numbers
	\item Numri pseudocasuali: pseudo random numbers
\end{itemize}
Una sequenza di numeri casuali è+ impredicible e perciò irriproducibile. Può essere generata da un processo fisico random, e.g. decadimento radioattivo, tempi di attivo di raggi cosmici.\\
Una sequenza di numeri pseudocasuali è ottenuta con una ben precisa formula matematica che fornisca una sequenza indistinguibile da una di numeri casuali.\\
Un tale generatore deve esserer:
\begin{itemize}
	\item Soggetto a test di randomness
	\item Avere un periodio (tipicamente $500 \cdot 10^6$ per rappresentazione a 32 bits)
\end{itemize}
I generatori pseudo-random attualmente più usati sono detti \textbf{congruenti moltiplicativi} e sono basati sulla formula:
$$n_i = a \cdot n_{i-1} (mod m)$$
Una sua variante, chiamata \textbf{congruenti misti} è basata sulla seguente formula:
$$n_i = a \cdot n_{i-1} + b (mod m)$$

In genere $m = 2^t$ con t numero di bits della rappresentazione di un intero sulla macchina considerata. \\
Il periodo è al massimo $m/4$.\\

\subsubsection{Metodo lineare congruente}
Questo metodo è basato sulla seguente formula ricorsiva:
$$n_{i+1} = (a\cdot n_i + c)mod(m)$$
Dove si sceglie un valore iniziale $n_0$ chiamato seed, e a,c sono scelti in maniera speciale. \\
In particolare $a,c > 0$ e $m > n_0,a,c$\\
\paragraph{Scelta del modulo m}
m deve essere il più grande possibile siccome il periodo non può eserer più grande di m.\\
Una sceglie solitamente m in modo da essere il più vicino possibile al più grande intero che si può rappresentare.\\
\paragraph{Scelta dei moltiplicatori, a}
E' stato provato da Greenberger che la sequenza avrà periodo m se e solo se: 
\begin{itemize}
	\item c è relativamente primo a m
	\item a-1 è un multiplo di p, per ogni primo p che divide m
	\item a-1 è un multiplo di 4 se m è un multiplo di 4
\end{itemize}
\subsubsection{Spiegazione del codice}
Prima di tutto implemento l'algoritmo per generare numeri casuali.\\
Scelgo di lavorare con modulo in 64bit. Decido di utilizzare come variabile un tipo \verb|long long int|, in questo modo nell'algoritmo ricorsivo posso porre il modulo a 0. Infatti volendo lavorare con modulo a 64bit, e siccome la variabile long long int occupa 64bit, non è necessaria la parte di calcolo del modulo in quanto è presente nella variabile stessa.
Definisco quindi la classe per generare numeri pseudocasuali.
\paragraph{Header pnrg.h}
\begin{verbatim}
#ifndef PRNG_H
#define PRNG_H

//definisco la variabile randtype. 
//unsigned (solo 0 o var positive), long long -> 64bit di memoria.
typedef unsigned long long int randtype;

class prng{

private:
randtype a;
randtype c;
randtype x;

public:
prng(); //constructor
~prng(); 

randtype randInt(); //numero casuale
double rand(double min=0., double max =1.); //numero casuale tra 0 e 1
};

#endif
\end{verbatim}
Per quanto detto prima l'operazione iterativa diventa meno complessa e viene implementata nel seguente modo.
\begin{verbatim}
#include "prng.h"
#include <climits>
#include <stdio.h>
#include <stdlib.h>
#include <time.h>


//Constructor 

prng::prng(){
a = 6364136223846793005U;
c = 1442695040888963407U;
x = 1.;
}


prng::~prng(){}
randtype prng::randInt(){

x = (a * x + c);
return x;
}

double prng::rand(double min, double max){

return (double)(max-min) * ((double) randInt()) /
((double) ULLONG_MAX ) + min;

}
\end{verbatim}

L'oggetto \texttt{pnrg} è inizializzato con i fattori $a$ e $c$ inizializzati con la stessa strategia della funzione \texttt{rand()} di c++. \\
Il metodo \texttt{randInt()} produce numeri casuali e la funzione \texttt{rand} produce numeri casuali nell'intervallo $[0,1]$





\end{document}